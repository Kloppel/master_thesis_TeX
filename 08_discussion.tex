\chapter{Conclusion and Outlook}
In this thesis, we provide alternative models for the previously used GPLVM for financial applications \cite{Nirwan_2019}. The added models include the Student-t version of the GPLVM, the SPLVM \ref{sec:student-t}, as well as the generalization of the GPLVM with both time dependent volatility (V-GPLVM) in section \ref{sec:v_gplvm}, as well as a time dynamic on the latent space of the GPLVM in section \ref{sec:td_gplvm}. The latter is then generalized to the T-SPLVM, the Student-t Process Latent Variable Model with a time dynamic on the latent space. All models are written down in terms of their mathematical formulation, and the generating process is laid out. After that, the application of these models using the stan \cite{stan_overview} language is described, where the variational bayes algorithm is used to evaluate the models. The main focus of this work is laid on the development of the models, which shows in the results chapter \ref{sec:results}. Here, also due to restraints of the size of calculations on the used computer cluster, the models including a time dynamic could only be solved with insufficiently high numbers of stock observations. The GPLVM and SPLVM model, in sections \ref{res:gplvm} and \ref{res:splvm} on the other hand were carried out with much higher numbers of parallel observations. \newline \newline
In the results section \ref{sec:results} the models are again introduced and their pitfalls are explained. At first, the GPLVM is shown, and the central systematic error in the reconstruction of the covariance structure is shown. Several sanity tests for the models are introduced and explained, which mainly consist of checking distributions of the inferred parameter values, comparing them to the expected values. Then, several other factors are taken into account, like the distributions of slope and intercept values of the stocks fit with Huber Regression, which directly relate to the observed systematical error. This systematical error is analyzed thoroughly, with the main factor being the number of stocks in the data matrix $Y$. A data set of returns normalized on volatility, using a HCMC optimized GARCH process, is also applied to the GPLVM, but fails due to problems that seem related to the GARCH process quality breaking down after half a year of trading days with observations. Afterwards the Student-t Process Latent Variable Model is applied to the same problem. The broader flanks of the Student-t distribution reduces the extent of the systematic error, but does not completely rid the model of it. The three models with a time dynamic are introduced afterwards, but they already fail during the sanity checks of the models, all while not being competitive with ELBO and coefficient of determination values. To make better assumptions about the models though, they would need to be compared not only to each other, but the GPLVM and SPLVM with entirely equal problem settings.
Putting the models into context, especially GPLVM and the SPLVM provide a high quality way of obtaining good covariance matrix estimates, that are necessary for e.g. market analysis, and even market predictions. \newline \newline
Further research and development into these models is needed, especially considering they can be used for all kings of regression and even classification tasks. The potential of the V-GPLVM, T-GPLVM and T-SPLVM, when applied with larger sizes of data matrices is also an untapped resource. As a statistical machine learning model these models have the potential to be applied usefully in different areas of research, since they have the potential to find correlation and infer the degree of causality. Also, when applied to financial markets, the sensible number of latent market variables, as they correspond to the number of latent dimensions of these models, can be approximated.